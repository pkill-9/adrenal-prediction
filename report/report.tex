\documentclass[a4paper]{article}

\usepackage[english]{babel}
\usepackage{blindtext}

\title{Supporting Seamless Access to Large Scale e-Infrastructures\\ 
    \large A Case
    Study in Use of High Performance Computing and Cloud-based Prediction
    for Diagnosis of Adrenal Tumour Types}
\author{Matthew Signorini \and Richard Sinnott}

\begin{document}

\begin{titlepage}
    \maketitle

    \begin{abstract}
        \blindtext
    \end{abstract}

    \tableofcontents
\end{titlepage}


\section{Introduction}

Computing is especially well suited to problems that require processing
of very large ammounts of data. Such problems range from analysis of the
human genome to the analysis of social media to isolate trends or make 
predictions; compute intensive problems. We will be looking closely at one 
example of this type of problem that is of interest to health 
professionals, and that is the classification
of the malignancy of adrenal tumour cases.

\subsection{Adrenal Tumours}
Adrenal tumours are estimated to be present in around 2\% of people, and
are almost always discovered incidentally, when a person has a CT scan
done for an unrelated reason. These tumours also become more common as a
person becomes older. With more people living to old age, and increasing
use of CT scans, doctors are required to diagnose and decide on appropriate
treatment for an increasing number of incidentally discovered adrenal
tumours. 

Of all adrenal tumour cases, only about 2 to 11\% are a malignant form
called adrenocortical carcinoma. These malignant tumours must be surgically
removed, however the majority of adrenal tumour cases may not require
urgent surgery, or may not require surgery at all. As we can see, there
is a need for a method to determine tumour malignancy that is appropriate
for the routine screening of a large number of patients.

One such screening method involves taking a urine sample, and measuring
the levels of certain products of steroid metabolism, exploiting the fact
that adrenal tumours have a different profile of steroid production
compared to healthy adrenal gland tissue. The steroid levels are measured
with GC-MS (gas chromatography followed by mass spectrometry), and
software is then used to assess malignancy.

Other techniqes that could be used include testing for markers that are 
found directly within
the tumour, which is able to reliably discriminate malignant tumours from
benign masses, but requires surgery to remove the adrenal mass
first. This defeats the aim of avoiding unneccessary surgery, making this
method unsuitable for routine screening. There is also a correlation between
tumour size and density and malignancy, size and density being easy to
measure with CT or
MRI, however these methods have been found to be significantly less
accurate than is desired. Another technique involves positron emmision
tomography, and has been found to have very high sensitivity and 
specificity, but has the disadvantage of being quite expensive.
Computational analysis of urine sample data by comparison is simple, non
invasive and inexpensive, making it the best suited for use in routine
screening.


\subsection{Machine Learning}
The task that the software component is faced with is referred to as
classification in the context of statistics and machine learning. Each
set of steroid levels from a patient constitutes an observation, or an
instance, and a classification algorithm seeks to identify which category,
malignant or benign, a new instance belongs to. The implementation of a
classification algorithm in software will be referred to as a classifier.

In machine learning terms, classification algorithms are supervised
learning algorithms. This means that they must be trained on a set of
labelled instances, being observations that are associated with the 
actual category that each observation belongs to. In this case, labelled
instances consist of steroid data associated with an actual diagnosis for
that patient.


\subsection{Performance of the Classifier}
The performance of the classifier was measured in a study involving a large
number of patients with adrenal tumours. In order to obtain labelled
training instances, each patient in the study was ultimately given a
malignant/benign diagnosis based on the evidence of metastasis or the
lack thereof for the two respective categories. The study reported that
they selected the most conclusively diagnosed patients for inclusion in
the training set, which would ensure that the classifier can give the
most reliable predictions possible since the learning algorithm would
otherwise try to fit potential misdiagnoses. The study also excluded
pregnant women and patients who were undergoing treatment with drugs such
as steroids, that alter steroid secretion in urine.


This project will explore the merits of two different models of large
scale computing for running this software: high performance computing and
cloud computing. We will also demonstrate how these two models can be used
together, in a system that can adapt it's capacity as required, completely
transparently to users.




% Adrenal tumours - prevalence, malignant/benign, outcomes for patients.
% Process for predicting malignancy, advantages over other methods.
% Analysis of the prediction algorithm.


\section{Overview of HPC}
% room full of servers, managed in house by an organisation
% edward at unimelb.
% Advantages -:
% Disadvantages
% Historical development from supercomputing?
\blindtext
\blindtext

\section{Overview of Cloud Computing}
% in this case, IaaS; we get access to a VM, someone else takes care of
% keeping it running, backups, maintenance of hardware.
% nectar, aws, azure
% iaas, paas, saas.
\blindtext
\blindtext

\section{System Architechture}
% batch of classifications is broken into two sets, some to run on HPC, the 
% rest on cloud.
\blindtext
\blindtext

\section{Front End}
\blindtext
\blindtext

\section{Performance}
% How much time when run sequentially? in parallel? also the time taken to
% spawn a new VM, which is not quick. This section will contain experimental
% design and results.
\blindtext
\blindtext

\section{Conclusion and Future Work}
\blindtext


\end{document}

% vim: ft=tex ts=4 sw=4 et
